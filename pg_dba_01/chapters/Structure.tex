\chapter{Install structure}\label{cha:INSTALLSTRUCT}

Depending on the installation method, the install structure is set up in a
single directory or in multiple folders.\newline

The install from source creates four subfolders in the target directory:
\textit{bin}, \textit{include}, \textit{lib} and \textit{share}.

\begin{itemize}

    \item \textbf{bin} contains the PostgreSQL binaries
    \item \textbf{include} contains the server's header files
    \item \textbf{lib} contains the shared libraries
    \item \textbf{share} contains the example files and the extension configurations

\end{itemize}

The packaged install puts the binaries and the libraries in the folder
/usr/lib/postgresql organised by major version. For example the 9.3 install
will put the binaries into /usr/lib/postgresql/9.3/bin and the libraries in
/usr/lib/postgresql/9.3/lib. The extensions and contributed modules are
installed into the folder /usr/share/postgresql with the same structure. The
Debian specific utilities and the symbolic link to the psql binary (which is at
/usr/lib/share/postgresql-common/pg\_wrapper) are stored in the directory
/usr/bin/. This file is a perl script which calls the PostgreSQL client reading
the version the cluster and the default database from the file ~/.postgresqlrc
or /etc/postgresql-common/user\_clusters.\newline

\section{The core binaries}

The PostgreSQL binaries can be split in two groups, the core and the wrappers
alongside with the contributed modules. Let's start then with the former group.

\subsection{postgres}\index{postgres}

\label{sec:POSTGRES}

This is the PostgreSQL's main process. The program can be started directly or
using the pg\_ctl utility. The second method is to be preferred as it offers a
simpler way to control the postgres process. Direct execution is the
unavoidable choice when the database won't start for an old XID near to the
wraparound failure\index{XID wraparound failure}. In this case the cluster can
only start in single user mode to perform a cluster wide vacuum. For historical
reasons there's also a symbolic link named postmaster pointing to the postgres
executable.

\subsection{pg\_ctl}\index{pg\_ctl}

\label{sub:PGCTL}

This utility is the simplest way for managing a PostgreSQL instance. The
program reads the postgres pid from the cluster's data area and sends the os
signals to start, stop or reload the process. It's also possible to send kill
signals to the running instance. pg\_ctl provides the following actions:

\begin{itemize}

    \item \textbf{init[db]} initialises a directory as PostgreSQL data area
    \item \textbf{start} starts a PostgreSQL instance
    \item \textbf{stop} shutdowns a PostgreSQL instance
    \item \textbf{reload} reloads the configuration's files
    \item \textbf{status} checks the PostgreSQL instance running status
    \item \textbf{promote} promotes a standby server
    \item \textbf{kill} sends a custom signal to the running instance

\end{itemize}

In chapter \ref{cha:MANAGING} we'll se how to manage the cluster.

\subsection{initdb}\index{initdb}

Is the binary which initialises the PostgreSQL data area. The directory to initialise must
be empty. Various options can be specified on the command line, like the character encoding or the
collation order.

\subsection{psql}\index{psql}

% TODO: Review statement 'The client it looks very essential'
This is the PostgreSQL command line client. The client it looks very essential, however is one of
the most flexible tools available to interact with the server and the only choice when working on
the command line.

\subsection{pg\_dump}\index{pg\_dump}

\label{sub:PGDUMP}

This is the binary dedicated to backup. Can produce consistent backups in
various formats. The usage is described shown in \ref{cha:BACKUP}.

\subsection{pg\_restore}\index{pg\_restore}

This program is used to restore a database reading a binary dump like the
custom or directory format. It's able to run the restore in multiple jobs in
order to speed up the process. The usage is described in \ref{cha:RESTORE}

\subsection{pg\_controldata}\index{pg\_controldata}\label{sub:PGCONTROLDATA}

This program can query the cluster's control file where PostgreSQL stores
critical information about the cluster activity and reliability.

\subsection{pg\_resetxlog}\index{pg\_resetxlog}

If a WAL file becomes corrupted the cluster cannot perform a crash recovery.
This lead to an unstartable cluster in case of system crash. In this
catastrophic scenario there's still a way to start the cluster. Using
pg\_resetxlog the cluster is cleared of any WAL file, the control file is
initialised from scratch and the transaction's count is restarted.\newline

The \textit{tabula rasa} comes with a cost indeed. The cluster loses any
reference between the transaction progression and the data files. The physical
integrity is lost and any attempt to run queries which write data will result
in corruption.\newline

The PostgreSQL's documentation is absolutely clear on this point.

\begin{verbatim}

After running pg_resetxlog the database must start without user access, the
entire content must be dumped, the data directory must be dropped and recreated
from scratch using initdb and then the dump file can be restored using psql or
pg_restore

\end{verbatim}

\section{Wrappers and contributed modules}

The second group of binaries is composed of the contributions and the wrappers.
The contributed modules add functions otherwise not available. The wrappers add
command line functions already present as SQL statements. Someone will notice
the lack of HA specific binaries like pg\_receivexlog and pg\_archivecleanup.
They have been purposely skipped because they are beyond the scope of this
book.

\subsection{The create and drop utilities}

The binaries with the prefix create and drop like, createdb createlang
createuser and dropdb, droplang, dropuser, are wrappers for the corresponding
SQL functions. Each program performs the creation and the drop action on the
corresponding named object. For example createdb adds a database to the cluster
and dropdb will drop the specified database.

\subsection{clusterdb}\index{clusterdb}

This program performs a database wide cluster on the tables with clustered
indices. The binary can run on a single table specified on the command line. In
\ref{sec:VACFULL} we'll take a look to CLUSTER and VACUUM FULL.

\subsection{reindexdb}\index{reindexdb}

The command does a database wide reindex. It's possible to run the command just
on a table or index passing the relation's name on the command line. In
\ref{sec:REINDEX} we'll take a good look at the index management.

\subsection{vacuumdb}\index{vacuumdb}

This binary is a wrapper for the VACUUM \index{VACUUM} SQL command. This is the
most important maintenance task and shouldn't be ignored. The program performs
a database wide VACUUM if executed without a target relation. Alongside with a
common vacuum it's possible to have the usage statistics updated on the same
time.

\subsection{vacuumlo}\index{vacuumlo}

This binary will remove the orphaned large objects from the pg\_largeobject
system table. The pg\_largeobject is used to store the binary objects bigger
than the limit of 1GB imposed by the bytea data type. The limit for a large
object it is 2 GB since the version 9.2. From the version 9.3 the limit was
increased to 4 TB.

\section{Debian's specific utilities}

Finally let's take a look at the Debian specific utilities. They are a
collection of perl scripts used to simplify the cluster's management. They are
installed in /usr/bin and mostly consist of symbolic links to the actual
executable. We already mentioned one of them in the chapter's introduction, the
psql pointing to the pg\_wrapper PERL script.

\subsection{pg\_createcluster}\index{pg\_createcluster}

This script adds a new PostgreSQL cluster with the given major version, if
installed, and the given name. The script puts all the configuration in
/etc/postgresql. Each major version has a dedicated directory within which is a
group of directories containing the cluster's specific configuration files. If
not specified the data directory is created in the folder /var/lib/postgresql.
It's possible to specify the options for initd.

\subsection{pg\_dropcluster}\index{pg\_dropcluster}

The program will delete a PostgreSQL cluster created previously with
pg\_createcluster. The program will not drop a running cluster. If the dropped
cluster has any tablespace those must be manually removed after the drop as the
program doesn't follow the symbolic links.

\subsection{pg\_lscluster}\index{pg\_lscluster}

Lists the clusters created with pg\_createcluster.

\subsection{pg\_ctlcluster}\index{pg\_ctlcluster}

\label{sub:PGCTLDEB}

% TODO: Understand this better to rewrite

The program manages the cluster in a similar way pg\_ctl does. Before version
9.2 this wrapper had dangerous behaviour for the shutdown. The script did not
offered a flexible way to provide the shutdown mode. More informations about
the shutdown sequence are in \ref{sec:SHUTDOWN_SEQ}. When run without any
arguments pg\_ctlcluster performs a smart shutdown mode. The --force option
tells the script to try a \textit{fast} shutdown mode. Unfortunately if the
database doesn't shutdown in a \textit{reasonable time} the script performs an
\textit{immediate} shutdown. After another short wait, if the the instance is
still up the script sends a \textit{kill -9} to the postgres process. Because
these kind of actions can result in data loss they should be done manually by
the DBA. It's better to avoid using pg\_ctlcluster for the shutdown.
