\chapter{PostgreSQL at a glance}

PostgreSQL is a first class product with enterprise class features. This
chapter is a general review on the product with a short section dedicated to
the database's history.

\section{A long time ago, in a galaxy far, far away...}

Following the works of the Berkeley's Professor Michael Stonebraker, in 1996
Marc G. Fournier\index{Marc G. Fournier} asked if there were any volunteers
interested in revamping the Postgres 95 project.\newline

\begin{smallverbatim}

Date: Mon, 08 Jul 1996 22:12:19-0400 (EDT)
From: ”Marc G. Fournier” <scrappy@ki.net>
Subject: [PG95]: Developers interested in improving PG95?
To: Postgres 95 Users <postgres95@oozoo.vnet.net>
Hi... A while back, there was talk of a TODO list and development moving forward on Postgres95 ...
at which point in time I volunteered to put up a cvs archive and sup server so that making updates
(and getting at the ”newest source code”) was easier to do...
... Just got the sup server up and running, and for those that are familiar with sup, the following
should work (ie. I can access the sup server from my machines using this):

..........................

\end{smallverbatim}

Bruce Momjian\index{Bruce Momjian},Thomas Lockhart\index{Thomas Lockhart}, and
Vadim Mikheev\index{Vadim Mikheev}, all replied to this email and formed the
very first PostgreSQL Global Development Team.\newline

Today, after almost 20 years and millions of rows of code, PostgreSQL is a
robust and reliable relational database. The most advanced open source
database. The slogan speaks truth indeed.

\section{Features}

Each time a new major release is released it adds new features to the already
rich feature set of PostgreSQL. What follows is a small excerpt of the
capabilities of the latest version of PostgreSQL.

\subsection{ACID compliant}

The word ACID is an acronym for Atomicity, Consistency, Isolation and
Durability. An ACID compliant database ensures those rules are enforced at all
times. \newline

\begin{itemize}

    \item The atomiticy is enforced when a transaction is ``all or nothing''. For example: If a
        transaction inserts a group of new rows and just one row violates the primary key then the entire
        transaction must be rolled back leaving the table as if nothing happened.

    \item The consistency ensures the database is constantly in a valid state. The database steps from
        one valid state to another valid state with no exceptions.

    \item The isolation is enforced when the database status can be reached as if all the concurrent
        transactions were run in serial.

    \item The durability ensures the committed transactions are saved on durable storage. In the event
        of a database crash the database must restore itself to the last valid state.

\end{itemize}

\subsection{MVCC}

PostgreSQL ensures atomiticy consistency and isolation using Multi Version
Concurrency Control (MVCC). The mechanism is incredibly efficient, it offers
great level of concurrency while keeping the transaction's snapshots isolated
and consistent. There is a single disadvantage in the implementation. We'll
see in detail in section \ref{sec:MVCC} how MVCC works and the reason why
there's no such thing as an update in PostgreSQL.

\subsection{Write ahead logging}

The durability is implemented in PostgreSQL using the write ahead log (WAL).
In short, when a data page is updated in the volatile memory the change is
saved immediately to a durable location, the write ahead
log\index{wal}\index{write ahead log}. The page is written to the corresponding
data file later. In the event of the database crash the write ahead log is
scanned and all the inconsistent pages are replayed on the data files. Each
segment size is usually 16 MB and they are automatically managed by PostgreSQL.
The write happens in sequence from the segment's start to the end. When this is
full PostgreSQL switches to a new one. When this happens there is a log switch.
\index{log switch}

\subsection{Point in time recovery}

\index{pitr}\index{point in time recovery}\index{log shipping}When PostgreSQL
switches to a new WAL this could be a new segment or a recycled one. If the old
WAL is archived in a safe location it's possible to get a copy of the physical
data files while the database is running.  The hot copy, alongside with the
archived WAL segments have all the informations necessary and sufficient to
recover the database's consistent state. The recovery by default terminates
when all the archived data files are replayed. It's possible to stop the
recovery at any given point in time.

\subsection{Standby server and high availability}

\index{standby server}\index{high availability}The standby server is a database
configured to stay in continuous recovery. This way a new archived WAL file is
replayed as soon as it becomes available. This feature was first in introduced
with  PostgreSQL 8.4 as warm standby\index{warm standby}. From version 9.0
PostgreSQL can be configured also as a hot standby\index{hot standby} which
allows connections for read only queries.

\subsection{Streaming replication}

The WAL archiving doesn't work in real time. The segment is shipped only after
a log switch and in a low activity server this can leave the standby behind the
master for a while. It's  possible to limit the problem using the
archive\_timeout parameter which forces a log swith after the given number of
seconds. However, using the streaming replication\index{streaming replication}
a standby server can get the wal blocks over a database connection in almost
real time. This feature allows the physical blocks to be transmitted over a
conventional database connection.

\subsection{Procedural languages}

PostgreSQL has many procedural languages. Alongside pl/pgsql it's possible to
write a procedure in many other popular languages like pl/perl and pl/python.
Anonymous functions are supported from version 9.1 onwards with the DO keyword.

\subsection{Partitioning}

Despite this the partitioning\index{partitioning}\index{constraint exclusion}
implementation in PostgreSQL is still very basic it's not complicated to build
an efficient partitioned structure using table inheritance.\newline

Unfortunately because the physical storage is distinct for each partition, it
is not possible to have a global primary key for the partitioned structure. The
foreign keys can be emulated in some way using the triggers.

\subsection{Cost based optimiser}

The cost based optimiser, or CBO,\index{cost based optimizer}\index{CBO} is one
of the PostgreSQL's points of strength. The execution plan is dynamically
determined from the data distribution and from the query parameters. PostgreSQL
also supports the genetic query optimizer GEQO.

\subsection{Multi platform support}

PostgreSQL\index{platform} supports almost any unix flavour, and from version
8.0 runs natively on Windows.

\subsection{Tablespaces}

The tablespace support permits a fine grained distribution of the data files
across filesystems. In sections \ref{sub:TBS-LOGICAL} and
\ref{sub:TBS-PHYSICAL} we'll see how to take advantage of this powerful
feature.

\subsection{Triggers}

The triggers are well supported on tables and views. A basic implementation of
event triggers is also present. The triggers can completely emulate the
updatable views feature.

\subsection{Views}

The read only views are well consodlidated in PostgreSQL. Version 9.3
introduced basic support for the materialised and updatable views. For the
materialised views there is no incremental refresh. The complex views, like
views joining two or more tables, cannot be updated.

\subsection{Constraint enforcement}

PostgreSQL supports primary keys and unique keys to enforce a table's data. The
referential integrity is guaranteed with foreign keys. We'll take a look to the
data integrity in chapter \ref{cha:DATAINT}.

\subsection{Extension system}

PostgreSQL the version 9.1 implements a very efficient extension system. The
command CREATE EXTENSION makes the installation of new features easy.

\subsection{Federated}

From PostgreSQL 9.1 is possible to have foreign tables pointing to external
data sources. PostgreSQL 9.3 also introduced the foreign table's write the
PostgreSQL's foreign data wrapper.
