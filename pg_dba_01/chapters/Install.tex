\chapter{Database installation}
\label{cha:DB_INSTALL}
In this chapter we will see how to install PostgreSQL on Debian Gnu Linux. We'll take a look at
two procedures, compiling from source and using the packages shipped by the pgdg
\index{pgdg apt repository} apt repository.\newline
There are advantages and disadvantages to both procedures. Compiling from source offers fine
grained control of all the aspects of the binaries configuration. It also doesn't have the risks of
unwanted restarts when upgrading and it's possible to install and upgrade the binaries without administrator privileges.\newline

The packaged install is easier to manage when deploying the new binaries to a server, in
particular if there is a large number of installations to manage. The binary packages are
released shortly after a new update is released. Because the frequency of the minor
releases is not fixed it could happen to have bugs for months.
For example: The bug which caused the standby server to crash when the master found invalid pages during
a conventional vacuum was fixed almost immediately. Unfortunately the release with the fix only
appeared after five months.\newline


\section{Install from source}
\label{sec:INSTSOURCE}
Using the configure script with the default settings requires the root access when installing.
That's because the permissions of the target location /usr/local don't allow writing by normal
users. This method adopts a different install location and requires the root access only for the os
user creation and the installation of the dependencies.
Before starting the PostgreSQL part ask your sysadmin to run the following commands.

\begin{itemize}

 \item useradd -d /home/postgres -s /bin/bash -m -U postgres
 \item passwd postgres
 \item apt-get update
 \item apt-get install build-essential  libreadline6-dev  zlib1g-dev
\end{itemize}

% apt-get build-dep postgresql
%
% build-dep
%     build-dep causes apt-get to install/remove packages in an attempt to
%     satisfy the build dependencies for a source package. By default the
%     dependencies are satisfied to build the package natively. If desired a
%     host-architecture can be specified with the --host-architecture option
%     instead.
%
% When run on a Vagrant box this installed the following:
%     build-essential debhelper dh-apparmor dpkg-dev g++ g++-4.6 gettext
%     html2text intltool-debian libcroco3 libdpkg-perl libgettextpo0
%     libstdc++6-4.6-dev libunistring0 make po-debconf

Please note the second step will require creating a new user password. Unless is a personal
test it's better to avoid obvious passwords like \textit{postgres}.\newline

In order to build the binaries we must download and extract the PostgreSQL's source tarball.

\begin{verbatim}
mkdir ~/download
cd ~/download
wget http://ftp.postgresql.org/pub/source/v9.3.5/postgresql-9.3.5.tar.bz2
tar xfj postgresql-9.3.5.tar.bz2
cd postgresql-9.3.5
\end{verbatim}


Using the configure script's --prefix option we'll point the install directory to a writable
location. We can also use a directory named after the the major version number. This will allow
us to have different PostgreSQL versions installed without problems.

\begin{verbatim}
mkdir -p /home/postgres/bin/9.3
./configure --prefix=/home/postgres/bin/9.3
\end{verbatim}

The configuration script will check all the dependencies and, if there's no error, will generate
the makefiles. Then we can start the build simply running the command \textit{make}. The time
required for compiling is variable and depends on the system's power. If you have a multicore
processor the make -j option can significantly improve the build time. When the build is complete
it's a good idea to to run the regression tests. Those tests are designed to find any regression or
malfunction before the binaries are installed.

\begin{verbatim}
make
<very verbose output>

make check

\end{verbatim}

The test's results are written in the source's subdirectory src/test/regress/results. If
there's no error we can finalise the installation with the command make install.

\begin{verbatim}
make install
\end{verbatim}



\section{Packaged install}
\label{sec:DEBIAN_INSTALL}

The PostgreSQL Global Development Group manages a repository in order to facilitate
installations on Debian based Linux distributions using Debian's packaging system.

Currently the supported distributions are

\begin{itemize}
 \item Debian 6.0 (squeeze)
 \item Debian 7.0 (wheezy)
 \item Debian unstable (sid)
 \item Ubuntu 10.04 (lucid)
 \item Ubuntu 12.04 (precise)
 \item Ubuntu 13.10 (saucy)
 \item Ubuntu 14.04 (trusty)
\end{itemize}

The PostgreSQL's versions available are
\begin{itemize}
 \item PostgreSQL 9.0
 \item PostgreSQL 9.1
 \item PostgreSQL 9.2
 \item PostgreSQL 9.3
 \item PostgreSQL 9.4
\end{itemize}

The packages are available for both amd64 and i386 architectures.

Anyway, the up to date list is available on the the wiki page
\href{http://wiki.postgresql.org/wiki/Apt}{http://wiki.postgresql.org/wiki/Apt}.\newline

All the installation steps require root privileges, via sudo or acquiring the root login via su.
Before starting to configure the repository it's a good idea to import the GPG key for the
package validation.

In a root shell simply run
\begin{verbatim}
wget --quiet -O - https://www.postgresql.org/media/keys/ACCC4CF8.asc | sudo apt-key add -
\end{verbatim}
When the key is imported create a file named pgdg.list into the directory /etc/apt/sources.d/ and
add the following row.

\begin{verbatim}
deb http://apt.postgresql.org/pub/repos/apt/ {codename}-pgdg main
\end{verbatim}


The distribution's codename can be found using the command lsb\_release -c.
e.g.
\begin{verbatim}
thedoctor@tardis:~$ lsb_release -c
Codename:       wheezy
\end{verbatim}

After the repository configuration the installation is completed with two simple commands.

\begin{verbatim}
apt-get update
apt-get install postgreql-9.3 postgreql-contrib-9.3 postgreql-client-9.3
\end{verbatim}

Be aware that this method, as automated installation task creates a new database cluster in
the default directory /var/lib/postgresql.

